% proposal document
%
% PROPOSAL: A brief plan of intention, strategy and accomplishments (about ½-1 page).
%
%----------------------------------------------------------------------------------------
%	PACKAGES AND OTHER DOCUMENT CONFIGURATIONS
%----------------------------------------------------------------------------------------

\section{Proposal}

In a research study, a university medical center urology group was interested in the association between prostate-specific antigen (PSA) and a number of prognostic clinical measurements in men with advanced prostate cancer. Data were collected on 97 men who were about to undergo radical prostectomies. The data given has identifications numbers, and provides information on 8 other variables on each person. The 8 variables being: PSA Level, Cancer Volume, Weight, Age, Benign Prostatic Hyperplasia, Seminal Vesicle Invasion, Capsular Penetration, and Gleason Score. \par
With this available data set, I will carry out a complete logistic regression analysis by first creating a binary response variable Y, called high-grade-cancer, by letting Y=1 if Gleason Score equals 8, and Y=0 otherwise (i.e., if Gleason Score equals 6 or 7). Thus, the response of interest is high-grade-cancer (Y), and the pool of predictors include those previously mentioned. \par
My analysis will consider transformations of predictors, the inclusion of second-order predictors, analysis of residuals and influential observations, model selection, goodness of fit evaluation, and the development of an ROC curve. Additionally, I will discuss the determination of a prediction rule for determining whether the grade of disease is predicted to be high grade or not, model validation, and finally asses the strengths and weaknesses of my final model. \\
