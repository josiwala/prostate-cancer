% literature review
%
% LITERATURE REVIEW: Mention any reference/example of similar/ related work/ article citation. Discuss briefly what has 
% been done in this problem/area of your interest (about 2-3 pages).
%

\section{Literature Review}

\subsection{Predictor Variables}

\noindent An understanding of the predictor variables in this particular study can be seen as follows:

\begin{itemize}

	\item \textbf{PSA Level}: Serum prostate-specific antigen level [mg/ml]. \par
		-Prostate cancer can often be found early by testing for prostate-specific antigen (PSA) levels in a man's blood. However, the PSA test is not 100\% accurate. \\
		- The chance of having prostate cancer increases as PSA level increases, but there is no set cutoff point that can tell for sure if a man does or does not have prostate cancer.
		
	\item \textbf{Cancer Volume}: Estimate of prostate cancer volume [cc]. \par
		-Studies have suggested that inflammation of the prostate gland (prostatitis) may be linked to an increased risk of prostate cancer, but other studies have not found such a link. \\
		-Inflammation is often seen in samples of prostate tissue that also contain cancer. The link between the two it not clear, and it remains an active area of research. 
		
	\item \textbf{Weight}: Prostate weight [gm]. \par
		-As related to cancer volume, studies have suggested that inflammation (and an increase is prostate weight) may be linked to an increased risk of prostate cancer. This relationship remains an active area of research.
		
	\item \textbf{Age}: Age of patient [years]. \par	
		-Prostate cancer is rare in men younger than 40, but the chance of having prostate cancer rises rapidly after age 50. About 6 in 10 cases of prostate cancer are found in men older than 65.
		
	\item \textbf{Benign Prostatic Hyperplasia}: Amount of benign prostatic hyperplasia [cm\textsuperscript{2}] \par
		-BPH is a term used to describe common, benign type of prostate enlargement caused by an increased number of normal prostate cells. This condition is more common as men get older and is not currently known to be linked to cancer.
		
	\item \textbf{Seminal Vesicle Invasion}: Presence of absence of seminal vesicle invasion: 1 if yes; 0 otherwise. \par
		-SVI is the presence of prostate cancer in the areolar connective tissue around the seminal vesicles and outside the prostate.
		
	\item \textbf{Capsular Penetration}: Degree of capsular penetration [cm]. \par
		-Cancer that has reached the outer wall of an organ (i.e. the prostate) is referred to as capsular penetration. Conversely, if cancer is strictly confined to the organ itself it is called organ-confined cancer.
		
	\item \textbf{Gleason Score}: Pathologically determined grade of disease using total score of two patterns (summed scores were either 6, 7, or 8 with higher scores indicating worse prognosis). \par
		-A measure of how likely the cancer is to grow and spread quickly. This is typically determined by the results of the prostate biopsy, or surgery.
		
\end{itemize}

\subsection{Related Research}

Doctors are still studying if screening tests will lower the risk of death from prostate cancer. The most recent results from two large studies show conflicting evidence, and unfortunately did not offer clear answers. \par

The outcomes of both studies can be summarized as follows:

\begin{itemize}

\item Early results from a large study done in the United States found that annual screening with PSA and DRE (digital rectal exam - for a DRE, the doctor puts a gloved, lubricated finger into the rectum to feel the prostate gland) did detect more prostate cancers than in men not screened, but this screening did not lower the death rate from prostate cancer. However, questions have been raised about this study, because some men in the non-screening group actually were screened during the study, which may have affected the results.

\item A European study did find a lower risk of death from prostate cancer with PSA screening (done about every 4 years), but the researchers estimated that roughly 781 men would need to be screened (and 27 cancers detected) to prevent one death from prostate cancer.

\item Neither of these studies has shown that PSA screening helps men live longer overall (i.e. lowers the overall death rate).

\end{itemize}

Prostate cancer is often slow-growing, so the effects of screening in these studies might become more clear in coming years. Also, both of these studies are being continued to see if a longer follow-up will give clearer results.


