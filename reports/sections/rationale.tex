% rationale
%
% RATIONALE: Why is this study important? What is the purpose of the study? What % do you want to achieve? (about ½-1 page).
%
%----------------------------------------------------------------------------------------
%	PACKAGES AND OTHER DOCUMENT CONFIGURATIONS
%----------------------------------------------------------------------------------------
% none


\section{Rationale}
Prostate Cancer is the most common cancer in American men. The American Cancer Society (ACS), a nationwide voluntary health organization, estimates 191,930 new cases of prostate cancer and over 33,000 deaths in year 2020 alone. Additionally, the typical cost of therapy to a prostate cancer patient is \$2,800/month after diagnosis (primarily from surgery and subsequently from office visits). A reliable and well understood testing/screening procedure needs to be in place to support early detection, and to minimize these current and unforgiving metrics. \par
Research suggests that prostate cancer typically begins as a pre-cancerous condition, and these conditions are sometimes found when a man has an invasive prostate biopsy (the removal of small pieces of the prostate to look for cancer.) If prostate cancer is found early as a result of \textit{screening}, it will probably be at an earlier and more treatable stage than if no screening were done. While this might seem like prostate cancer screening would always be a good thing, there are still issues surrounding screening procedures that make it unclear if the benefits outweigh the risks for most men. \par
For example, the popular PSA screening test is not 100\% accurate. This test can sometimes have abnormal results even when a man does not have cancer (false-positive result), or normal results when a man does have cancer (false-negative result). Consequently, false-positive results can lead to some men to get prostate biopsies (with risks of pain, infection, and bleeding) when they do not have cancer, and false-negative results can give men a false sense of security even though they may actually have cancer. \par
Another important issue is that even if screening does detect prostate cancer, doctors often cannot tell if the cancer is truly dangerous and needs to be treated. Prostate cancer can grow so slowly that it may never cause a man problems in his lifetime, and some men who seek screening may be diagnosed with a prostate cancer that they would have never known about otherwise. It would never have led to their death, or even cause any symptoms. Finding a "disease" like this that would never cause problems is known as \textbf{overdiagnosis}. \par
The problem with overdiagnosis in prostate cancer is that many of the men might still be treated with either surgery or radiation, either because the doctor cannot be sure how quickly the cancer might grow or spread, or the man is uncomfortable knowing he has cancer and is not receiving any treatment. The treatment of a cancer that would never have caused any problems is known as \textbf{overtreatment}, and the major downsides after surgery or radiation may include urinary, bowel, and/or sexual side effects that can seriously affect a man's quality of life. Thus, men and their doctors often struggle to decide if treatment is needed, or if the cancer can just be closely watched without being treated right away. Even when men are not treated right away, they still need regular blood PSA tests and prostate biopsies to determine if their need for treatment in the future. \par
For now, the ACS recommends that men thinking about getting tested for prostate cancer learn as much as they can so they can make informed decisions based on available information, discussions with their doctors, and their own views on the possible benefits, risks, and limits of prostate cancer screening. To combat and better navigate these difficulties, research needs to continue to grow the understanding of prostate cancer, and to build stronger predictive models which can improve the outlook of male lives, and also alleviate undo strain on the healthcare system.